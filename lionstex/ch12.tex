%
% The Lion's Commentary, file ch12.tex, version 1.5, 17 May 1994
%
\se{Traps and System Calls}

This chapter is concerned with the way
the system handles traps in general and
system calls in particular.

There are quite a number of conditions
which can cause the processor to
``trap''. Many of these are quite
clearly error conditions, such as
hardware or power failures, and UNIX
does not attempt any sophisticated
recovery procedures for these.

The initial focus for our attention is
the principal procedure in the file
``trap.c''.

\sbs{trap (2693)}

The way that this procedure is invoked
was explored in Chapter Ten. The
assembler ``trap'' routine carries out
certain fundamental housekeeping tasks
to set up the kernel stack, so that
when this procedure is called, everything appears to be kosher.

The ``trap'' procedure can operate as
though it had been called by another
``C'' procedure in the normal way with
seven parameters

\begin{center}
 dev, sp, rl, nps, r0, pc, ps.
\end{center}

(There is a special consideration which
should be mentioned here in passing.
Normally all parameters passed to ``C''
procedures are passed by value. If the
procedure subsequently changes the
values of the parameters, this will not
affect the calling procedure directly.

However if ``trap'' or the interrupt
handlers change the values of their
parameters, the new values will be
picked up and reflected back when the
``previous mode'' registers are
restored.)

The value of ``dev'' was obtained by capturing the value of the processor
status word immediately after the trap
and masking out all but the lower five
bits. Immediately before this, the processor status word had been set using
the prototype contained in the
appropriate vector location.

Thus if the second word of the vector
location was ``br7+n;'' (e.g. line 0516)
then the value of ``dev'' will be n.

\bd
\item[2698:] ``savfp'' saves the floating point
 registers (for the PDP11/40, this
 is a no-op!);

\item[2700:] If the previous mode is ``user
 mode', the value of ``dev'' is
 modified by the addition of the
 octal value 020 (2662);

\item[2701:] The stack address where r0 is
 stored is noted in ``u.u\_ar0'' for
 future reference. (Subsequently
the various register values can
be referenced as ``u.u\_ar0[Rn]''.);

\item[2702:] There is now a multi-way ``switch''
 depending on the value of ``dev''.
\ed

At this point we can observe that UNIX
divides traps into three classes,
depending on the prior processor mode
and the source of the trap:

\bd
\item[(a)] kernel mode;

\item[(b)] user mode, not due to a ``trap''
 instruction;

\item[(c)] user mode, due to a ``trap''
 instruction.
\ed


\sbs{Kernel Mode Traps}

The trap is unexpected and with one
exception, the reaction is to ``panic''.
The code executed is the ``default'' of
the ``switch'' statement:

\bd
\item[2716:] Print:

\bi
\item the current value of the seventh
kernel segment address register
(i.e. the address of the current
per process data area);

\item the address of ``ps'' (which is in
the kernel stack); and

\item the trap type number;
\ei

\item[2719:] ``panic'', with no return.
\ed

Floating point operations are only used
by programs, and not by the operating
system. Since such operations on the
PDP11/45 and 11/70 are handled asynchronously, it is possible that when a
floating point exception occurs, the
processor may have already switched to
kernel mode to handle an interrupt.

Thus a kernel mode floating point
exception trap can be expected occasionally and is the concern of the
current user program.

\bd
\item[2793:] Call ``psignal'' (3963) to set a
 flag to show that a floating
 point exception has occurred;

\item[2794:] Return.
\ed

This raises an interesting question: ``Why are the kernel mode
and user mode floating point
exceptions handled slightly di�ferently?''


\sbs{User Mode Traps}

Consider first of all a trap which is
not generated as the result of the execution of a ``trap instruction''. This
is regarded as a probable error for
which the operating system makes no
provision apart from the possibility of
a ``core dump''. However the use program
itself may have anticipated it and provided for it.

The way this provision is made and
implemented is the subject of the next
chapter. At this stage, the principal
requirement is to ``signal'' that the
trap has occurred.

\bd
\item[2721:] A bus error has occurred while
 the system is in user mode. Set
 ``i'' to the value ``SIGBUS'' (0123);

\item[2723:] The ``break'' causes a branch out
 of the ``switch'' statement to line
 2818;

\item[2733:] Apart from the one special case
noted, the treatment of illegal
instructions is the same at this
level as for bus errors;

\item[2739:]
\item[2743:]
\item[2747:]
\item[2796:] Cf. the comment for line 2721.
\ed


Note that cases ``4+USER'' (power fail)
and\\
``7+USER'' (programmed interrupt) are
handled by the ``default'' case (line
2715).

\bd
\item[2810:] This represents a case where
 operating system assistance is
 required to extend the user mode
 stack area.

The assembler routine ``backup''
(1012) is used to reconstruct the
situation that existed before
execution of the instruction that
caused the trap.

``grow'' (4136) is used to do the
actual extension.
\ed

The procedure ``backup'' is non-trivial
and its comprehension involves a careful consideration of various aspects of
the PDP11 architecture. It has been
left for the interested reader to pursue privately.

As noted for the PDP11/40, ``backup'' may
not always succeed because the processor does not save enough information to
resolve all possibilities.

\bd
\item[218:] Call ``psignal'' (3963) to set the
 appropriate ``signal''. (Note that
 this statement is only reached
 from those cases of the ``switch''
 which included a ``break'' statement.);

\item[2821:] ``issig'' checks if a ``signal'' has
 been sent to the user program,
 either just now or at some earlier time and has not yet been
 attended to;

\item[2822:] ``psig'' performs the appropriate
 actions. (Both ``issig'' and ``psig''
 are discussed in detail in the
 next chapter.);

\item[2823:] Recalculate the priority for the
 current process.
\ed

\sbs{System Calls}


User mode programs use ``trap'' instructions as part of the ``system call''
mechanism to call upon the operating
system for assistance.

Since there are many possible ``versions'' of the ``trap'' instruction, the
type of assistance requested can be and
is encoded as part of the ``trap''
instruction.


Parameters which are part of a system
call may be passed from the user program in different ways:

\bd
\item[(a)] via the special register r0;

\item[(b)] as a set of words embedded in
 the program string following the
 ``trap'' instruction;

\item[(c)] as a set of words in the
 program's data area. (This is
 the ``indirect'' call.)
\ed


Indirect calls have a higher overhead
than direct system calls. Indirect
calls are needed when the parameters
are data dependent and cannot be determined at compile time.

Indirect calls may sometimes be avoided
if there is only one data dependent
parameter, which is passed via r0. In
choosing which parameters should be
passed via r0, the system designers
have presumably been guided by their
own experience, since the pattern
doesn't satisfy the law of least astonishment.

The ``C'' compiler does not give special
recognition to system calls, but treats
them in the same way as other procedures. When the loader comes to
resolve undetermined references, it
satisfies these with library routines
which contain the actual ``trap''
instructions.

\bd
\item[2752:] The error indicators are reset;

\item[2754:] The user mode instruction which
 caused the trap is retrieved and
 all but the least significant six
 bits are masked off. The result
 is used to select an entry from
 the array of structures,
``sysent''. The address of the
selected entry is stored in
``callp'';


\item[2755:] The ``zeroeth'' system call is the
``indirect'' system call, in which
the parameter passed is actually
the address in the user program
data space of a system call
parameter sequence.
\ed

Note the separate uses of ``fuword'' and
``fuiword''. The distinction between
these is unimportant on the PDP11/40,
but is most important on machines with
separate ``i'' and ``d'' address spaces;

\bd
\item[2760:] ``i=077'' simulates a call on the
very last system call (2975),
which results in a call on
``nosys'' (2855), which results in
an error condition which will
usually be fatal for the user
mode program;

\item[2762:]
\item[2765:] The number of arguments specified
in ``sysent'' is the actual number
provided by the user programmer,
or that number less one if one
argument is transferred via r0.
The arguments are copied from the
user data or instruction area
into the five element array
``u.u\_arg''. (From ``sysent'' (Sheet
29) it would seem that four elements would have been sufficient
for ``u\_area[ ]'' -- is this an
allowance for future inflation?);

\item[2770:] The value of the first argument
is copied into ``u.u\_dirp'', which
seems to function mainly as a
convenient temporary storage
location;

\item[2771:] ``trapl'' is called with the
 address of the desired system
 routine. Note the comment beginning on line 2828;

\item[2776:] When an error occurs, the ``c-bit''
in the old processor status word
is set (see line 2658) and the
error number is returned via r0.
\ed

\sbs{System Call Handlers}

The full set of system calls may be
reviewed in the file ``sysent.c'' on
Sheet 29, but more relevantly, these
are discussed in full detail in Section
II of the UPM.

The procedures which handle the system
calls are found mostly in the files
``sysl.c'', sys2.c``, sys3.c'' and
``sys4.c''.

Two important ``trivial'' procedures are
``nullsys'' (2855) and ``nosys'' (2864)
which are found in the file ``trap.c''.


\sbs{The File `sysl.c'}

This file contains the procedures for
five system calls, of which three will
be considered now, and two (``rexit'' and
``wait'') will be deferred to the next
chapter.

The first procedure in this file, and 
also the first system call we have
encountered, is ``exec''.

\sbs{exec (3020)}

This system call, \#11, changes a process executing
one program into a process executing a different program.
See Section ``EXEC(II)'' of the UPM.
This is the longest and one of the most
important system calls.

\bd
\item[3034:] ``namei'' (6618) (which is discussed in detail in Chapter 19)
converts the first argument
(which is a pointer to a character
 string defining the name of 
 the new program) into an ``inode''
 reference. (``inodes'' are essential parts of the file
 referencing mechanism.);

\item[3037:] Wait if the number of ``exec''s
 currently under way is too large
 (See the comment on line 3011.);

\item[3040:] ``getblk(NODEV)'' results in the
 allocation of a 512 byte buffer
 from the pool of buffers. This
 buffer is used temporarily to
 store in core, that information
 which is currently in the user
 data area, and which is needed to
 start the new program. Note that
 the second argument in ``u.u\_arg''
 is a pointer to this information;

\item[3041:] ``access'' returns a non-zero
result if the file is not executable. The second
condition examines whether the file is a
directory or a special character file.
(It would seem that by making
this test earlier, e.g. just
after line 3036, the efficiency
of the code could be improved.);

\item[3052:] Copy the set of arguments from
 the user space into the temporary
 buffer;

\item[3064:] If the argument string is too
large to fit in the buffer, take
an error exit;

\item[3071:] If the number of characters in
the argument string is odd, add
an extra, null character;

\item[3090:] The first four words (8 bytes) of
 the named file are read into
 ``u.u\_arg''. The interpretation of
 these words is indicated in the
 comment beginning on line 3076
 and, more fully, in the section
 ``A.OUT(V)'' of the UPM.

Note the setting of ``u.u\_base'',
``u.u\_count'', ``u.u\_offset'' and
``u.u\_segflg'' preparatory to the
read operation;

\item[3095:] If the text segment is not to be
protected, add the text area size
to the data area size, and set the former to
zero;

\item[3105:] Check whether the program has a
``pure'' text area, but the program
file has already been opened by
some other program as a data
file. If so, take the error exit;

\item[3127:] When this point is reached, the
 decision to execute the new program is irrevocable i.e. there is
 no longer the opportunity to
 return to the original program
 with an error flag set;

\item[3129:] ``expand'' here actually implies a
major contraction, to the ``per
process data'' area only;

\item[3130:] ``xalloc'' takes care of allocating
(if necessary) and linking to the
text area;

\item[3158:] The information stored in the
 buffer area is copied into the
 stack in the user data area of
 the new program;

\item[3186:] The locations in the kernel stack
which contain copies of the ``previous'' values of the registers in
user mode are set to zero, except
for r6, the stack pointer, which
was set at line 3155;

\item[3194:] Decrement the reference count for
 the ``inode'' structure;

\item[3195:] Release the temporary buffer;

\item[3196:] Wake up any other process waiting
 at line 3037.
\ed


\sbs{fork (3322)}

A call on ``exec'' is frequently preceded
by a call on ``fork''. Most of the work
for ``fork'' is done by ``newproc'' (1826),
but before the latter is called, ``fork''
makes an independent search for a slot
in the ``proc'' array, and remembers the
place as ``p2'' (3327).

``newproc'' also searches ``proc'' but
independently. Presumably it always
locates the same empty slot as ``fork'',
since it does not report the value
back. (Why is there no confusion on
this point?)

\bd
\item[3335:] For the new process, ``fork''
returns the value of the parent's
process identification, and initialises various accounting
parameters;

\item[3344:] For the parent process, ``fork''
returns the value of the child's
process identification, and {\bf skips}
the user mode program counter by
one word.
\ed

\noindent Note that the values finally returned
to a ``C'' program are slightly different
from the above. Refer to the section
FORK(II) of the UPM.

\sbs{sbreak (3354)}

This procedure implements system call
\#17 which is described in the Section
``BREAK (II)'' of the UPM. The comment at
the head of the procedure has confused
more than one reader: clearly the identifier
``break'' is used in ``C'' programs
(leave an enclosing program loop) in an
entirely different way from that
intended here (change the size of the
program data area).

``sbreak'' has clear similarities with
the procedure ``grow'' (4136) but unlike
the latter, it is only invoked explicitly and
may in fact cause a contraction of the data area as well as an
expansion (depending on the new desired
size).

\bd
\item[3364:] Calculate the new size for the
 data area (in 32 word blocks);

\item[3371:] Check that the new size is consistent with the memory
segmentation constraints;

\item[3376:] The area is shrinking. Copy the
 stack area down into the former
 data area. Call ``expand'' to trim
 off the excess;

\item[3386:] Call ``expand'' to increase the
 total area. Copy the stack area
 up into the new part, and clear
the areas which were formerly
occupied by the stack.
\ed

The following procedures which are also
contained in ``sysl.c'' are described in
Chapter 13:

\begin{verbatim}
  rexit (3205)     wait (3270)
  exit  (3219)
\end{verbatim}

\sbs{The Files `sys2.c' and `sys3.c'}

``sys2.c'' and ``sys3.c'' are mainly concerned with the file system and
input/output, and they have been
relegated to Section Four of the
operating system source code.


\sbs{The File `sys4.c'}

All the procedures in this file implement system calls.
The following procedures are described in Chapter 13:

\begin{verbatim}
    ssig (3614)     kill (3630)
\end{verbatim}

The following procedures are straightforward and have been left for the
amusement and edification of the reader:

\begin{verbatim}
    getswit (3413) sync (3486)
    gtime (3420)   getgid (3472)
    stime (3428)   getpid (3480)
    setuid (3439)  nice (3493)
    getuid (3452)  times (3656)
    setgid (3460)  profil (3667)
\end{verbatim}

The following procedures which are concerned with file systems, are described
later:

\begin{verbatim}
    unlink (3510)   chown (3575)
    chdir (3538)    smdate (3595)
    chmod (3560)
\end{verbatim}
